\documentclass{beamer}
\mode<presentation>
\usetheme[compress]{Berlin}
\usecolortheme{beaver}
\setbeamertemplate{navigation symbols}{}
\setbeamertemplate{footline}{}
\usepackage{hyperref}
\hypersetup{linkcolor=}

\title{Novelty Detection in SkyMapper}
\subtitle{Developing an Open-World Classifier}
\institute{Australian National Univerity}
\author[Alasdair Tran]{Alasdair Tran \\
	\texttt{u4921817@anu.edu.au}}
\date{13 March 2015}
\setbeamertemplate{headline}{} % hide header

\begin{document}
	
\begin{frame}
	\titlepage
\end{frame}

% % % % % % % % % % % % % % % % % % % % % % % % % % % % % % % % % % % % % % % % % % % % % 
\section{Motivation}
\subsection{}
\begin{frame}{What Is This?}
	\begin{center}
	\includegraphics<1>[height=7cm]{images/oldest_star}
	\end{center}
\end{frame}

\subsection{}
\begin{frame}{SkyMapper Survey}
	\begin{columns}[T]
	  	\begin{column}{.5\textwidth}
  			\begin{itemize}
  				\item A comprehensive survey of the entire southern sky.
  				\item Huge amount of data.
  				\item Need an automatic large-scale classifier.
  			\end{itemize}
	  	\end{column}
	  	\begin{column}{.5\textwidth}
		  	\includegraphics<1>[height=4cm]{images/skymapper}
	  	\end{column}
	\end{columns}
\end{frame}


% % % % % % % % % % % % % % % % % % % % % % % % % % % % % % % % % % % % % % % % % % % % % 
\section{Project}
\subsection{}
\begin{frame}{What Computers Actually See}
	\begin{align*}
		\begin{pmatrix}
		18.34817 \\ 
		16.99189 \\ 
		16.51436 \\ 
		16.28606 \\ 
		16.10466 \\ 
		18.22762 \\ 
		16.84973 \\ 
		16.33935 \\ 
		16.09515 \\ 
		15.96056 \\ 
		1.545161
		\end{pmatrix}
		\uncover<2->{
		\implies
		\includegraphics[height=3cm, trim=0 10cm 0 -10cm]{images/black_box}}
		\uncover<3->{
		\implies
		\text{Star}}
	\end{align*}
\end{frame}

\begin{frame}{Classification Task}
	$$\text{Photometric Measurements} \implies \text{Classifier} \implies \text{Class Prediction}$$
\end{frame}


\begin{frame}{Easy Problem}
	Easy if we have info about all classes during the training phase.
	\begin{center}
		\includegraphics<1>[height=6cm]{images/posterior}
	\end{center}
\end{frame}


\begin{frame}{Example Training Data}
Subset of the Sloan dataset (30,000 data points and 3 classes).
	\begin{center}
		\includegraphics<1>[height=7cm]{images/sdss_pca}
	\end{center}
\end{frame}

\begin{frame}{Training the Classifier}
	One approach: Fit a Gaussian distribution to each class.
	\begin{center}
		\includegraphics<1>[height=7cm]{images/sdss_pca_qda}
	\end{center}
\end{frame}



\begin{frame}{Detecting Unknown Classes}
	Are these the best decision boundaries?
	\begin{center}
		\includegraphics<1>[height=7cm]{images/sdss_pca_unknown}
	\end{center}
\end{frame}


\begin{frame}{Further Information}
	My project is hosted at \href{https://github.com/alasdairtran/mclass-sky}{\texttt{github.com/alasdairtran}}.
	\vspace{2em}
	
	My supervisors:
	\begin{itemize}
		\item Cheng Soon Ong, NICTA
		\item Justin Domke, NICTA
		\item Christian Wolf, RSAA 
	\end{itemize}
\end{frame}




\end{document}
