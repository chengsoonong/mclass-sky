%%
%% Template intro.tex
%%

\chapter{Introduction}
\label{cha:intro}
The night sky has always been a source of wonder throughout history.
The Ancient Greeks were the first to bring 
mathematics to astronomy and sought a rational explanation of celestial bodies.
But we could only discover so much with our naked eyes. It was not until the invention
of new technology during the Scientific Revolution like the telescope that we began to
make significantly more progress. And yet, 400 years after Galileo made the first working
telescope, most of the sky still remains unknown to us.

This, however, is starting to change. The Sloan Digital Sky Survey (SDSS) began
in 2000 and has since obtained 800 million sets of photometric measurements
and 3 million spectra of the northern sky. Similar projects are going on
to map the southern sky including the VST ATLAS and SkyMapper projects.
 We now face another challenge of how
to analyse this huge amount of data. Fortunately, we can bring machine learning
to astronomy and let the computer to do the pattern recognition task.
This thesis is an attempt to study a small part of the interaction between these
two disciplines.

% % % % % % % % % % % % % % % % % % % % % % % % % % % % % % % % % % % % % % % % % % % % % % % % % %
\section{Contributions}
\label{sec:contributions}
The thesis centres around the problem of how we can spectroscopically label objects in
the most efficient way and letting the computer do the rest of the classification with only
photometric data. Our key novel contribution is the application of Thompson sampling,
a Bayesian solution to the exploration-exploitation trade-off, to the selection of six
active learning heuristics. As far as we know, this is the first time that such active learning
heuristics are applied to the astronomical domain. It is also the first time that the
multi-arm bandit problem with Thompson sampling is used in the heuristic selection context.
Along the way, we make four other contributions that would support the active learning
experiment. These are
	\begin{itemize}
		\item Applying three competing sets of extinction
		vectors on photometric measurements and see if such correction has any effect on the accuracy rate.
		\item Identifying good photometric features using both well-known machine learning techniques
		like polynomial transformation and domain knowledge such as the use of colours.
		\item Applying various classifiers to do photometric classification and studying the
		pros and cons of each one.
		\item Deriving an extension of the
		posterior balanced accuracy rate to the multi-class setting, which will be used to evaluate
		the performance of our algorithms.
	\end{itemize}
An important part of this project involves creating an
open-source, extendable, and well-documented Python package that allows astronomers to perform active learning
routines and make quick visualisations of photometric data. The package and
reproducible code of all experiments are available on the project's GitHub repository\footnote{
	\url{https://github.com/alasdairtran/mclearn}}.


% % % % % % % % % % % % % % % % % % % % % % % % % % % % % % % % % % % % % % % % % % % % % % % % % %
\section{Thesis Outline}
\label{sec:orgnisation}
The thesis consists of four main chapters, two of which are dedicated to a discussion of ideas and
the other two focus on experiments:
	\begin{itemize}
		\item Chapter \ref{cha:astro} introduces the reader to the tools that astronomers use
		to map the sky. This includes important concepts like photometry, magnitudes, dust
		extinction, and the celestial coordinate system. We also give a brief overview of
		two datasets, SDSS and VST ATLAS, that are used in our experiments.
		
		\item Chapter \ref{cha:ml} surveys the relevant literature in pool-based active learning. This leads
		to  a discussion of the multi-arm bandit problem and the use of Thompson sampling to a
		setting where each bandit arm is an active learning heuristic. We end the chapter with a 
		derivation of multi-class posterior balanced accuracy rate, an important
		performance measure for data with unbalanced classes.
		
		\item Chapter \ref{cha:expt1} and \ref{cha:expt2} test the ideas discussed so far in a
		series of experiments on the SDSS and the VST-ATLAS datasets. A detailed protocol and a
		thorough discussion of the results are given for each experiment. In particular, we offer
		insights into the areas where Thompson sampling outperforms random sampling and where it
		underperforms. 
	\end{itemize}
For those interested in reproducing the experiments and applying the ideas in this thesis
to their own datasets, the Appendix includes a guide on how to install and use the accompanying
Python package \texttt{mclearn}. We also provide information on how to obtain the SDSS dataset
with SQL queries.


%%% Local Variables: 
%%% mode: latex
%%% TeX-master: "thesis"
%%% End: 
