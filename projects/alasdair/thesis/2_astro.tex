%%
%% Template intro.tex
%%

\chapter{Mapping the Night Sky}
\label{cha:astro}
When many people think about astronomy, the first things that pop into their mind are
breathtaking images of the sky like the famous Hubble Deep Field. But pretty images alone
are not enough to do real science. For us to make any meaningful observations about
astronomical objects, we need actual quantitative measurements.

\section{Spectroscopy and Photometry}
\label{sec:spec}

One such quantitative approach is spectroscopy. This involves
using diffraction grating to disperse light and measure the amount of electromagnetic radiation,
or flux, emitted from an object at small wavelength intervals. As the name implies, we end
up with a spectrum, like the one shown in Figure \ref{fig:vega}. The shape of the spectrum
and its absorption lines allow us to deduce many useful properties such as the object's
temperature and chemical composition.

Unfortunately, it can be very costly to take high resolution spectra of faint objects since
we are spreading light thinly across many wavelengths. Photometry gets around this problem
by separating light into fewer groups and thus reducing the wavelength resolution
\cite{romanishin02}. Specifically we have a set of filters, each of which can be put in
front of the CCD
camera to allow only light from certain wavelengths to pass through. Associated with
each filter is a transmission function $T(\lambda)$ that tells us
the fraction of light that the filter will transmit at wavelength $\lambda$.
Figure \ref{fig:vega} shows $T(\lambda)$ of the five bandpasses (u, g, r, i, and z) that
are used in the SDSS.

\begin{figure}[tbp]
	\centering
	\includegraphics[width=\textwidth]{figures/2_astro/vega_filters_and_spectrum}
	\caption[Spectrum of the star Vega and the ugriz bandpasses]{The black curve
		is the spectrum of Vega, the fifth brightest star in the night sky. The spectrum
		tells us how much radiation Vega emmits at each wavelength. We also show five
		transmission functions, one for each of the five ugriz filters. The transmission
		function tells us how much light can get through the filter at each wavelength.}
	\label{fig:vega} \index{Vega}
\end{figure}


\section{Measuring Fluxes}
\label{sec:mag}
When light hits the CCD, all we have initially are counts of photons, one for each pixel.
The first challenge is to assign the photons to distinct objects. Two models are used
in the SDSS, depending on whether we assume the object is an extended source or a point source.


\subsection{Petrosian Flux}
Galaxies are extended-source objects so we need to define an aperture radius, within which
all the photons are added together to obtain the flux of the galaxy. Since galaxies have
poorly defined edges, a consistent method to pick the aperture radius is required.
\shortciteN{blanton01} define a quantity called the Petrosian ratio:
	\begin{IEEEeqnarray*}{lCl}
		\mathfrak{R}_p(r) = \frac{\int_{0.8r}^{1.25r} 2\pi s I(s) \, ds}{\pi (1.25^2 - 0.8^2) r^2}
							\bigg/ \frac{\int_0^r 2\pi s I(s) \, ds}{\pi r^2}
	\end{IEEEeqnarray*}
This is the ratio of the mean local surface brightness over an annulus at $r$ to the mean
surface brightness within $r$. The quantity $I(r)$ is the galaxy surface brightness profile
and can be estimated from the photon count. With this, we define the Petrosian
radius $r_p$ as the radius such that $\mathfrak{R}_p(r_p) = 0.2$. This is one of the features
in the SDSS dataset that will prove to be very useful in distinguishing galaxies from
point-source objects. The aperture radius is then chosen to be $2r_p$ to ensure that almost
all of the light from a typical galaxy is captured. Finally for each bandpass, we can calculate
the corresponding Petrosian flux as
	\begin{IEEEeqnarray*}{lCl}
		f &=& \int_{0}^{2 r_p} 2 \pi s I(s) \, ds
	\end{IEEEeqnarray*}

\subsection{Point Spread Function Fitting}
Stars, quasars, and white dwarfs are unresolved point sources so they can be modelled by a point
spread function (PSF). This approach is particularly useful when we examine a dense region
like a globular cluster. In such places, given the amount of overlap, it would be very
difficult define an aperture that includes only photons from an object and excludes all others
from the neighbours. In PSF fitting, we assume that all objects have the same shape, which allows
us to fit a Gaussian model to each of them. We then iteratively vary the position and flux
of the objects until the model produces the observed light distribution \cite[Chapter 10]{palmer01}.
The flux estimated by the converged model is called the PSF flux.


\section{Magnitudes and Colours}
\subsection{Apparent Magnitudes}
The fluxes of the brightest and the dimmest objects in the sky can differ by many orders of
magnitudes. This motivates us to take one step further and convert fluxes to inverse
hyperbolic sine (or arsinh) apparent magnitudes:
	\begin{IEEEeqnarray*}{lCl}
		m &=& -\frac{2.5}{\ln(10)} \Bigg[ \arsinh\bigg(\frac{f/f_0}{2b}\bigg) + \ln(b) \Bigg]
	\end{IEEEeqnarray*}
Here $f_0$ is the flux of the object with a conventional magnitude of 0 and $b$ is the
softening parameter. A nice feature of this magnitude system is that for bright objects
with a high signal-to-noise ratio, it behaves like a logarithmic scale, i.e. with
every decrease of 1 in the magnitude scale, the object becomes 2.5 times brighter.\footnote{
	The reader might wonder why the scale works in reverse, with a small magnitude
	corresponding to more brightness. This is the convention created two centuries ago by
	the Greek astronomer Hipparchus, which, for better or worse, has stuck with us ever since.}
At the same time, as the flux tends toward zero for fainter objects, the arsinh function
(unlike the log function) ensures that the magnitudes are still well-defined \cite{lupton99}.

\subsection{Absolute Magnitudes and Colours}
\label{sub:colours}

The problem with apparent magnitudes is their dependence on distance. Objects that are further
away form us are fainter and hence have higher apparent magnitudes. Of course, being
further away does not change the fact that an object is a star or a galaxy. Thus if we want
to study their intrinsic properties, we need to remove this dependency. One method is
to convert to the absolute magnitude, which is defined as the object's apparent magnitude if
it were exactly 10 parsecs away from Earth:
	\begin{IEEEeqnarray*}{lCl}
		M &=& m - 5 \log_{10} \bigg(\frac{d}{10}\bigg)
	\end{IEEEeqnarray*}
This conversion requires the knowledge of $d$, the actual distance of an object in parsecs.
In practice, it is often difficult to estimate $d$, so we resort to an easier method of taking
the difference between the amount of light received in two bandpasses. For example,
the $u -g$ colour is calculated as
	\begin{IEEEeqnarray*}{lCl}
		m_{u-g} &=& m_u - m_g \\
		        &=& M_u + 5 \log_{10} \bigg(\frac{d}{10}\bigg) -
		            M_g + 5 \log_{10} \bigg(\frac{d}{10}\bigg) \\
		        &=& M_u - M_g
	\end{IEEEeqnarray*}
When taking the difference, the conversion factor cancels and the colour of an object no longer
depends on its distance.


\section{Equatorial Coordinate System}
Imagine a very large celestial sphere with Earth at its centre. By projecting onto the
inside surface of this sphere, we have a way to specify the position
of any astronomical object. In this thesis, we use the equatorial coordinate system,
where an object's position is specified by two numbers, a right ascension (ra) and a declination
(dec). Figure \ref{fig:mollweide} shows the Mollweide projection of the celestial sphere.
We will use this map throughout the thesis to visualise some results.

\begin{figure}[tbp]
	\centering
	\includegraphics[width=\textwidth]{figures/2_astro/mollweide_map}
	\caption[Mollweide projection of the celestial sphere]{This is the Mollweide projection
		of the celestial sphere under the equatorial coordinate system.
		The red line indicates the plane of the Milky Way. To avoid cluttering, the coordinate labels will not be shown on later maps.}
	\label{fig:mollweide} \index{Mollweide projection}
\end{figure}

The equatorial coordinate system, like any other systems, needs to have a reference point.
Here anything on the celestial sphere that is
directly above the Earth's equator will have a declination of 0\deg. To define a zero point
for the right ascension, we use the fact that the centre of the Sun passes through the plane of the
Earth's equator twice a year. The first crossing point usually happens on 21 March and is called
the vernal equinox. We now define the right ascension of the vernal equinox to be
0\deg~\cite[Chapter~1]{sparke07}.\footnote{
	There is actually a slight complication. Since the Earth's rotation axis is not
	fixed due to precession, the ra-dec coordinates of an object relative to the origin will
	actually change slowly over time. Thus we need to also fix a time in which the
	coordinates are measured. In the SDSS, 1 January 2000 12:00 Terrestial Time is chosen as
	a reference point.}




\section{Datasets}
\label{sec:datasets}
Two datasets, SDSS and VST ATLAS, are used in our experiments. Below we give a quick overview
of what each dataset contains. For more information on how to obtain them, refer to
Appendix \ref{cha:datasets}.

\subsection{SDSS Dataset}
\begin{figure}[tbp]
	\centering
	\includegraphics[width=\textwidth]{figures/4_expt1/map_prediction_forest_all}
	\caption[Coverage of the SDSS]{The distribution of the 800 million scanned objects
		in the survey: Note that the coverage is not uniform. A darker colour
		corresponds to more objects being scanned in a particular area.}
	\label{fig:coverage}
\end{figure}

The main dataset in our investigation comes from the SDSS, a comprehensive survey of the
Northern Sky that began operation in 2000. This dataset consists of 800 million objects, covering
about a third of the sky \cite{alam15}. Figure \ref{fig:coverage} shows the coverage of the survey.

For each object, we are given 11 features:
	\begin{itemize}
		\item The PSF magnitude in each of the five ugriz bands.
		\item The Petrosian magnitude in each of the five ugriz bands.
		\item The Petrosian radius in the r-band.
	\end{itemize}
Only 2.8 million out of the 800 million objects have been spectroscopically classified into three
classes: galaxies, quasars, and stars. From Figure \ref{fig:class_dist_sdss}, we can see that the
number of galaxies is twice that of stars and three times that of quasars. This leads to
the problem of class imbalance. A few approaches on how to minimise any bias toward
the dominant class during classification will be discussed in chapter \ref{cha:ml}.

\begin{figure}[tbp]
	\centering
	\begin{subfigure}{.5\textwidth}
		\centering
		\includegraphics[width=0.99\textwidth]{figures/2_astro/sdss_class_distribution}
		\caption{SDSS Dataset}
		\label{fig:class_dist_sdss}
	\end{subfigure}%
	\begin{subfigure}{.5\textwidth}
		\centering
		\includegraphics[width=0.99\linewidth]{figures/2_astro/vstatlas_class_distribution}
		\caption{VST ATLAS Dataset}
		\label{fig:class_dist_vst}
	\end{subfigure}
	\caption[Distribution of the classes in the SDSS and VST ATLAS datasets]{
		Class distribution of the labelled objects in the SDSS and VST ATLAS datasets: Observe
		that both datasets exhibit the problem of class imbalance. Most objects in the
		SDSS are galaxies, while stars are the dominant class in the VST ATLAS.}
	\label{fig:class_dist}
\end{figure}


\subsection{VST ATLAS Dataset}
A second and much smaller dataset comes from a more recent survey, the VST ATLAS. This project
aims to survey 4500 deg$^2$ of the Southern Sky to roughly the same depth as the SDSS
\cite{shanks15}.
There are 35,000 objects in total which have been classified by an expert into four classes
this time: stars, galaxies, quasars, and white dwarfs. For each object, we are provided
with 7 features:
	\begin{itemize}
		\item The calibrated magnitude from the r-band.
		\item Four colour indices using the ugriz filters: $u-g$, $g-r$, $r-i$, and $i-z$.
		\item Two colour indices in the infrared: $r-W1$ and $W1-W2$.
	\end{itemize}
The two infrared channels $W1$ and $W2$ are measured by the Wide-field Infrared Survey Explorer
(WISE), one of NASA's space telescopes. These channels will prove quite useful in distinguishing
between classes. As we can see from Figure \ref{fig:class_dist_vst}, the problem of class
imbalance is even worse in this dataset. For instance, we have 25 times more stars than white
dwarfs. Since research is still on-going, the coordinates of the objects are not yet publicly
available.



\section{Dust Extinction}
Throughout the Milky Way we have interstellar dust which can absorb and scatter light as it
travels from its source to us. The scattering is especially strong at shorter wavelengths,
Since less of the blue light arrives on Earth, an object will appear redder than it actually
is.

This would not be a big problem if this reddening process were uniform throughout the
celestial sphere, since all the magnitudes and colours will simply shift by a constant
term. Indeed, this is the case with the VST-ATLAS, since the field where the objects were
surveyed exhibits no drift in the reddening.
However, the SDSS covers a much larger portion of the sky. As shown in
Figure \ref{fig:reddening}, the reddening varies by quite a bit throughout the field. This
mean we might need to correct the SDSS magnitudes for dust extinction.

\begin{figure}[tbp]
	\centering
	\includegraphics[width=\textwidth]{figures/2_astro/ebv_map}
	\caption[Galactic reddening map in the SDSS]{Galactic reddening $E(B-V)$ map in the SDSS:
		A darker colour indicates a greater amount of reddening.}
	\label{fig:reddening}
\end{figure}

There are three competing sets of reddening corrections. In all of the three sets, we
start with the reference reddening quantity
	\begin{IEEEeqnarray*}{lCl}
		E(B-V) &=& \frac{\A_r}{2.751}
	\end{IEEEeqnarray*}
where the values of $\A_r$ are included in the SDSS dataset.

The most popular correction set for a long time was created by \citeN{schlegel98}.
The correction values for each of the five passbands can be
calculated as follows:
	\begin{IEEEeqnarray*}{lCl}
		\A_u &=& 5.155 \cdot E(B-V) \\
		\A_g &=& 3.793 \cdot E(B-V) \\
		\A_r &=& 2.751 \cdot E(B-V) \\
		\A_i &=& 2.086 \cdot E(B-V) \\
		\A_z &=& 1.479 \cdot E(B-V)
	\end{IEEEeqnarray*}
Later, \citeN{schlafly11} applied a different extinction curve, giving us the following
correction values:
 	\begin{IEEEeqnarray*}{lCl}
 		\A_u &=& 4.239 \cdot E(B-V) \\
 		\A_g &=& 3.303 \cdot E(B-V) \\
 		\A_r &=& 2.285 \cdot E(B-V) \\
 		\A_i &=& 1.698 \cdot E(B-V) \\
 		\A_z &=& 1.263 \cdot E(B-V)
 	\end{IEEEeqnarray*}
Recently, \citeN{wolf14} remapped the $E(B-V)$ scale to
	\begin{IEEEeqnarray*}{lCl}
		E'(B-V) &=&
		\begin{cases}
			E(B-V) & \text{if } E(B-V) \in [0, 0.04], \\
			E(B-V) + 0.5(E(B-V) - 0.04) & \text{if } E(B-V) \in [0, 0.08], \\
			E(B-V) + 0.02 & \text{if } E(B-V) \in [0.08, +\infty].
		\end{cases}
	\end{IEEEeqnarray*}
which can then be used to calculate a new set of correction values:
	\begin{IEEEeqnarray*}{lCl}
 		\A_u &=& 4.305 \cdot E'(B-V) \\
 		\A_g &=& 3.288 \cdot E'(B-V) \\
 		\A_r &=& 2.261 \cdot E'(B-V) \\
 		\A_i &=& 1.714 \cdot E'(B-V) \\
 		\A_z &=& 1.263 \cdot E'(B-V)
	\end{IEEEeqnarray*}
Each of these correction values can then be subtracted from the corresponding magnitudes
to make up for the loss of the scattered light. Let us call these three reddening correction
sets SDF98, SF11, and W14, respectively. As part of our experiments, we will compare them and
see which one offers the best results.







%%% Local Variables: 
%%% mode: latex
%%% TeX-master: "thesis"
%%% End: 
