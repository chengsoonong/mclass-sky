%%
%% Template appendix.tex
%%

\appendix

\chapter{How to Obtain the Datasets}
\label{cha:datasets}

Testing the display of SQL code:

\begin{minted}[fontsize=\footnotesize, frame=single, tabsize=4]{sql}
SELECT
	p.ra, p.dec,
	CASE s.class WHEN 'GALAXY' THEN 'Galaxy'
				 WHEN 'STAR' THEN 'Star'
				 WHEN 'QSO' THEN 'Quasar'
				 END AS class,
	s.subclass,
	s.z AS redshift,
	s.zErr AS redshiftErr,
	s.zWarning,
	p.psfMag_u, p.psfMagErr_u,
	p.psfMag_g, p.psfMagErr_g,
	p.psfMag_r, p.psfMagErr_r,
	p.psfMag_i, p.psfMagErr_i,
	p.psfMag_z, p.psfMagErr_z,
	p.petroMag_u, p.petroMagErr_u,
	p.petroMag_g, p.petroMagErr_g,
	p.petroMag_r, p.petroMagErr_r,
	p.petroMag_i, p.petroMagErr_i,
	p.petroMag_z, p.petroMagErr_z,
	p.extinction_u, p.extinction_g, p.extinction_r,
	p.extinction_i, p.extinction_z,
	p.petroRad_r, p.petroRadErr_r

FROM PhotoObj AS p
	LEFT JOIN SpecObj AS s
	ON s.bestobjid = p.objid
\end{minted}


\chapter{Guide to Using mclearn}
\label{cha:mclearn}

\section{Installation}
\label{sec:installation}

\section{Usage and Examples}
\label{sec:usage}


\chapter{Vectorisation of the Variance Estimation}
\label{cha:vectorise}

In estimating the variance of the unlabelled pool, there are two matrices we wish to compute...




%%% Local Variables: 
%%% mode: latex
%%% TeX-master: "thesis"
%%% End: 
