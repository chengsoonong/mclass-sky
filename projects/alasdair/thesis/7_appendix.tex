%%
%% Template appendix.tex
%%

\appendix

\chapter{How to Obtain the Datasets}
\label{cha:datasets}

The following SQL query is used to get the main SDSS dataset containing 2.8 million
labelled objects from the Sloan SkyServer\footnote{
	url{http://skyserver.sdss.org/CasJobs/}}

\begin{minted}[fontsize=\footnotesize, frame=single, tabsize=4]{sql}
SELECT
	-- right ascension and declination in degrees
	p.ra, p.dec,
	
	-- class of object, expert opinion (galaxy, star, or quasar)
	CASE s.class WHEN 'GALAXY' THEN 'Galaxy'
				 WHEN 'STAR' THEN 'Star'
				 WHEN 'QSO' THEN 'Quasar'
				 END AS class,
	
	s.subclass, -- subclass of object
	
	-- redshift of object from spectrum with error, expert opnion
	s.z AS redshift,
	s.zErr AS redshiftErr,
	s.zWarning,
	
	-- PSF and Petrosian mags in 5 bands (ugriz) with error
	p.psfMag_u, p.psfMagErr_u,
	p.psfMag_g, p.psfMagErr_g,
	p.psfMag_r, p.psfMagErr_r,
	p.psfMag_i, p.psfMagErr_i,
	p.psfMag_z, p.psfMagErr_z,
	
	p.petroMag_u, p.petroMagErr_u,
	p.petroMag_g, p.petroMagErr_g,
	p.petroMag_r, p.petroMagErr_r,
	p.petroMag_i, p.petroMagErr_i,
	p.petroMag_z, p.petroMagErr_z,
	
	-- extinction values
	p.extinction_u, p.extinction_g, p.extinction_r,
	p.extinction_i, p.extinction_z,
	
	-- size measurement in r-band in arc seconds
	p.petroRad_r, p.petroRadErr_r

FROM PhotoObj AS p
JOIN SpecObj AS s
ON s.bestobjid = p.objid

WHERE
	-- only include objects with complete and reasonably accurate data
	p.psfMagErr_u BETWEEN 0 AND 3
	AND p.psfMagErr_g BETWEEN 0 AND 3
	AND p.psfMagErr_r BETWEEN 0 AND 3
	AND p.psfMagErr_i BETWEEN 0 AND 3
	AND p.psfMagErr_z BETWEEN 0 AND 3
	AND p.petroMagErr_u BETWEEN 0 AND 3
	AND p.petroMagErr_g BETWEEN 0 AND 3
	AND p.petroMagErr_r BETWEEN 0 AND 3
	AND p.petroMagErr_i BETWEEN 0 AND 3
	AND p.petroMagErr_z BETWEEN 0 AND 3
	AND p.petroRadErr_r BETWEEN 0 AND 3
	AND s.zErr BETWEEN 0 AND 0.1
	AND s.zWarning = 0    -- spectrum is ok
\end{minted}


The following query is used to extract photometric measurements from all 800 million
in the database. Since the file is fairly big (around 200 GB in size),
a special request might need to be made.

\begin{minted}[fontsize=\footnotesize, frame=single, tabsize=4]{sql}
SELECT
	p.ra, p.dec,
	CASE s.class WHEN 'GALAXY' THEN 'Galaxy'
				 WHEN 'STAR' THEN 'Star'
				 WHEN 'QSO' THEN 'Quasar'
				 END AS class,
	s.subclass,
	s.z AS redshift,
	s.zErr AS redshiftErr,
	s.zWarning,
	p.psfMag_u, p.psfMagErr_u,
	p.psfMag_g, p.psfMagErr_g,
	p.psfMag_r, p.psfMagErr_r,
	p.psfMag_i, p.psfMagErr_i,
	p.psfMag_z, p.psfMagErr_z,
	p.petroMag_u, p.petroMagErr_u,
	p.petroMag_g, p.petroMagErr_g,
	p.petroMag_r, p.petroMagErr_r,
	p.petroMag_i, p.petroMagErr_i,
	p.petroMag_z, p.petroMagErr_z,
	p.extinction_u, p.extinction_g, p.extinction_r,
	p.extinction_i, p.extinction_z,
	p.petroRad_r, p.petroRadErr_r
FROM PhotoObj AS p
LEFT JOIN SpecObj AS s
ON s.bestobjid = p.objid
\end{minted}






\chapter{Guide to Using mclearn}
\label{cha:mclearn}

[To do: a quick two page tutorial on main features on the python package \texttt{mclearn}]

\section{Installation}
\label{sub:installation}

\section{Usage and Examples}
\label{sub:usage}


%\chapter{Vectorisation of the Variance Estimation}
%\label{cha:vectorise}

%In estimating the variance of the unlabelled pool, there are two matrices we wish to compute...



\chapter{Dust Extinction Vectors} \index{dust extinction}
\label{cha:dustvectors}

The SDF98 extinction values are given in the SDSS dataset. To calculate the other
two extinction vectors, we start with a reference reddening quantity
\begin{IEEEeqnarray*}{lCl}
	E_{B-V} &=& \frac{\A_r}{2.751}
\end{IEEEeqnarray*}
where $\A_r$ is the SDF98 extinction value in the r-band.

As a check, we can actually recover the SDF98 extinction vector as follows:
\begin{IEEEeqnarray*}{lCl}
	\A_u &=& 5.155 \cdot E_{B-V} \\
	\A_g &=& 3.793 \cdot E_{B-V} \\
	\A_r &=& 2.751 \cdot E_{B-V} \\
	\A_i &=& 2.086 \cdot E_{B-V} \\
	\A_z &=& 1.479 \cdot E_{B-V}
\end{IEEEeqnarray*}
Later, \citeN{schlafly11} applied a different extinction curve, giving us the following
correction values:
\begin{IEEEeqnarray*}{lCl}
	\A_u &=& 4.239 \cdot E_{B-V} \\
	\A_g &=& 3.303 \cdot E_{B-V} \\
	\A_r &=& 2.285 \cdot E_{B-V} \\
	\A_i &=& 1.698 \cdot E_{B-V} \\
	\A_z &=& 1.263 \cdot E_{B-V}
\end{IEEEeqnarray*}
Recently, \citeN{wolf14} remapped the $E_{B-V}$ scale to
\begin{IEEEeqnarray*}{lCl}
	E'_{B-V} &=&
	\begin{cases}
		E_{B-V} & \text{if } E_{B-V} \in [0, 0.04], \\
		E_{B-V} + 0.5(E_{B-V} - 0.04) & \text{if } E_{B-V} \in [0, 0.08], \\
		E_{B-V} + 0.02 & \text{if } E_{B-V} \in [0.08, +\infty].
	\end{cases}
\end{IEEEeqnarray*}
which can then be used to calculate a new set of correction values:
\begin{IEEEeqnarray*}{lCl}
	\A_u &=& 4.305 \cdot E'_{B-V} \\
	\A_g &=& 3.288 \cdot E'_{B-V} \\
	\A_r &=& 2.261 \cdot E'_{B-V} \\
	\A_i &=& 1.714 \cdot E'_{B-V} \\
	\A_z &=& 1.263 \cdot E'_{B-V}
\end{IEEEeqnarray*}
Each of these correction values need to be subtracted from the corresponding magnitudes
to make up for the loss of the scattered light.





\chapter{Supplementary Results}
\label{cha:supp}

In our experiments, we generated many plots. In this Appendix, we put together results
that are not vital to the main narrative but still somewhat interesting.

\section{Reliability of Probabilities Estimates}
\label{sec:forest_prob}

We found that the probabilities estimated by both random forests and multinomial regression
to be unreliable. [To do: include learning curves of random forest, one-vs-rest, and
multinomial regression, and warm-start here, test both SDSS and VST ATLAS]

\begin{figure}[p]
	\centering
	\begin{subfigure}{\textwidth}
		\centering
		\includegraphics[width=\textwidth]{figures/appendix/sdss_forest_multinom}
		\caption{Recall map of galaxies.}
		\label{fig:sdss_forest_multinom}
	\end{subfigure}\\
	\begin{subfigure}{\textwidth}
		\centering
		\includegraphics[width=\linewidth]{figures/appendix/vstatlas_forest_multinom}
		\caption{Recall map of stars.}
		\label{fig:vstatlas_forest_multinom}
	\end{subfigure}
	\caption{Recall map of when there is no corrections.}
	\label{fig:forest_multinom}
\end{figure}


\section{Effects of Dust Extinction on Recall}

In Chapter \ref{cha:expt1} we tested the effect of three different extinction vectors 
on the accuracy rate. \ref{fig:map_recall_uncorrected} show how recall rate is distributed
over the celestial sphere. Overall, the recall on galaxies is almost perfect, while
the recall on stars is fairly average. On the three pages after that, Figures \ref{fig:map_recall_sfd98}, \ref{fig:map_recall_sf11}, and \ref{fig:map_recall_w14} show
the improvement on recall after each extinction vector is applied. The interesting bit
is that there is a patch of stars right next to the Milky Way plane that gets a big improvement
in recall after reddening correction. Thus the extinction vector would be very important
if there were more objects that are closer to the Milky Way plane.


\begin{figure}[p]
	\centering
	\begin{subfigure}{\textwidth}
		\centering
		\includegraphics[width=0.75\textwidth]{figures/appendix/map_recall_uncorrected_Galaxy}
		\caption{Recall map of galaxies.}
		\label{fig:map_recall_uncorrected_galaxies}
	\end{subfigure}\\
	\begin{subfigure}{\textwidth}
		\centering
		\includegraphics[width=0.75\linewidth]{figures/appendix/map_recall_uncorrected_Star}
		\caption{Recall map of stars.}
		\label{fig:map_recall_uncorrected_stars}
	\end{subfigure}
	\begin{subfigure}{\textwidth}
		\centering
		\includegraphics[width=0.75\linewidth]{figures/appendix/map_recall_uncorrected_Quasar}
		\caption{Recall map of quasars.}
		\label{fig:map_recall_uncorrected_quasars}
	\end{subfigure}
	\caption{Recall map of when there is no corrections.}
	\label{fig:map_recall_uncorrected}
\end{figure}


\begin{figure}[p]
	\centering
	\begin{subfigure}{\textwidth}
		\centering
		\includegraphics[width=0.75\textwidth]{figures/appendix/map_recall_sfd98_Galaxy}
		\caption{Recall improvement map of galaxies.}
		\label{fig:map_recall_sfd98_galaxies}
	\end{subfigure}\\
	\begin{subfigure}{\textwidth}
		\centering
		\includegraphics[width=0.75\linewidth]{figures/appendix/map_recall_sfd98_Star}
		\caption{Recall improvement map of stars.}
		\label{fig:map_recall_sfd98_stars}
	\end{subfigure}
	\begin{subfigure}{\textwidth}
		\centering
		\includegraphics[width=0.75\linewidth]{figures/appendix/map_recall_sfd98_Quasar}
		\caption{Recall improvement map of quasars.}
		\label{fig:map_recall_sfd98_quasars}
	\end{subfigure}
	\caption{Recall improvement map of when the SFD98 correction set is applied.}
	\label{fig:map_recall_sfd98}
\end{figure}


\begin{figure}[p]
	\centering
	\begin{subfigure}{\textwidth}
		\centering
		\includegraphics[width=0.75\textwidth]{figures/appendix/map_recall_sf11_Galaxy}
		\caption{Recall improvement map of galaxies.}
		\label{fig:map_recall_sf11_galaxies}
	\end{subfigure}\\
	\begin{subfigure}{\textwidth}
		\centering
		\includegraphics[width=0.75\linewidth]{figures/appendix/map_recall_sf11_Star}
		\caption{Recall improvement map of stars.}
		\label{fig:map_recall_sf11_stars}
	\end{subfigure}
	\begin{subfigure}{\textwidth}
		\centering
		\includegraphics[width=0.75\linewidth]{figures/appendix/map_recall_sf11_Quasar}
		\caption{Recall improvement map of quasars.}
		\label{fig:map_recall_sf11_quasars}
	\end{subfigure}
	\caption{Recall improvement map of when the SF11 correction set is applied.}
	\label{fig:map_recall_sf11}
\end{figure}


\begin{figure}[p]
	\centering
	\begin{subfigure}{\textwidth}
		\centering
		\includegraphics[width=0.75\textwidth]{figures/appendix/map_recall_w14_Galaxy}
		\caption{Recall improvement map of galaxies.}
		\label{fig:map_recall_w14_galaxies}
	\end{subfigure}\\
	\begin{subfigure}{\textwidth}
		\centering
		\includegraphics[width=0.75\linewidth]{figures/appendix/map_recall_w14_Star}
		\caption{Recall improvement map of stars.}
		\label{fig:map_recall_w14_stars}
	\end{subfigure}
	\begin{subfigure}{\textwidth}
		\centering
		\includegraphics[width=0.75\linewidth]{figures/appendix/map_recall_w14_Quasar}
		\caption{Recall improvement map of quasars.}
		\label{fig:map_recall_w14_quasars}
	\end{subfigure}
	\caption{Recall improvement map of when the W14 correction set is applied.}
	\label{fig:map_recall_w14}
\end{figure}


\section{Total Number of Selection of Heuristics in Thompson Sampling}

Finally, Figures \ref{fig:sdss_sigmas} and \ref{fig:vstatlas_sigmas} show how...

\begin{figure}[p]
	\centering
	\begin{subfigure}{.5\textwidth}
		\centering
		\includegraphics[width=\textwidth]{figures/5_thompson/vstatlas_bl_no_selections}
		\caption{Balanced pool and logistic regression}
		\label{fig:sdss_bl_no_selections}
	\end{subfigure}%
	\begin{subfigure}{.5\textwidth}
		\centering
		\includegraphics[width=\linewidth]{figures/5_thompson/sdss_br_no_selections}
		\caption{Balanced pool and RBF SVM}
		\label{fig:sdss_br_no_selections}
	\end{subfigure}
	\begin{subfigure}{.5\textwidth}
		\centering
		\includegraphics[width=\textwidth]{figures/5_thompson/sdss_ul_no_selections}
		\caption{Unbalanced pool and logistic regression}
		\label{fig:sdss_ul_no_selections}
	\end{subfigure}%
	\begin{subfigure}{.5\textwidth}
		\centering
		\includegraphics[width=\linewidth]{figures/5_thompson/sdss_ur_no_selections}
		\caption{Unbalanced pool and RBF SVM}
		\label{fig:sdss_ur_no_selections}
	\end{subfigure}
	\caption[Variance of heuristics (SDSS)]{
		Variance of the expected reward in Thompson sampling with the SDSS dataset.}
	\label{fig:sdss_no_selections}
\end{figure}

\begin{figure}[p]
	\centering
	\begin{subfigure}{.5\textwidth}
		\centering
		\includegraphics[width=\textwidth]{figures/5_thompson/vstatlas_bl_no_selections}
		\caption{Balanced pool and logistic regression}
		\label{fig:vstatlas_bl_no_selections}
	\end{subfigure}%
	\begin{subfigure}{.5\textwidth}
		\centering
		\includegraphics[width=\linewidth]{figures/5_thompson/vstatlas_br_no_selections}
		\caption{Balanced pool and RBF SVM}
		\label{fig:vstatlas_br_no_selections}
	\end{subfigure}
	\begin{subfigure}{.5\textwidth}
		\centering
		\includegraphics[width=\textwidth]{figures/5_thompson/vstatlas_ul_no_selections}
		\caption{Unbalanced pool and logistic regression}
		\label{fig:vstatlas_ul_no_selections}
	\end{subfigure}%
	\begin{subfigure}{.5\textwidth}
		\centering
		\includegraphics[width=\linewidth]{figures/5_thompson/vstatlas_ur_no_selections}
		\caption{Unbalanced pool and RBF SVM}
		\label{fig:vstatlas_ur_no_selections}
	\end{subfigure}
	\caption[Variance of heuristics (VST ATLAS)]{
		Variance of the expected reward in Thompson sampling with the VST ATLAS dataset.}
	\label{fig:vstatlas_no_selections}
\end{figure}


\section{Variance of the Reward Function in Thompson Sampling}

Finally, Figures \ref{fig:sdss_sigmas} and \ref{fig:vstatlas_sigmas} show how $\sigma^2$,
the variance of the expected reward, changes with the training size. As we would expect,
the variance decreases exponentially over time. This is expected since as we increase
the training size, we become more certain of the classifier's accuracy. In addition, the incremental change
of the accuracy rate shrinks over time as well as we approach an accuracy of 1.

\begin{figure}[p]
	\centering
	\begin{subfigure}{.5\textwidth}
		\centering
		\includegraphics[width=\textwidth]{figures/5_thompson/vstatlas_bl_sigmas}
		\caption{Balanced pool and logistic regression}
		\label{fig:sdss_bl_sigmas}
	\end{subfigure}%
	\begin{subfigure}{.5\textwidth}
		\centering
		\includegraphics[width=\linewidth]{figures/5_thompson/sdss_br_sigmas}
		\caption{Balanced pool and RBF SVM}
		\label{fig:sdss_br_sigmas}
	\end{subfigure}
	\begin{subfigure}{.5\textwidth}
		\centering
		\includegraphics[width=\textwidth]{figures/5_thompson/sdss_ul_sigmas}
		\caption{Unbalanced pool and logistic regression}
		\label{fig:sdss_ul_sigmas}
	\end{subfigure}%
	\begin{subfigure}{.5\textwidth}
		\centering
		\includegraphics[width=\linewidth]{figures/5_thompson/sdss_ur_sigmas}
		\caption{Unbalanced pool and RBF SVM}
		\label{fig:sdss_ur_sigmas}
	\end{subfigure}
	\caption[Variance of heuristics (SDSS)]{
		Variance of the expected reward in Thompson sampling with the SDSS dataset.}
	\label{fig:sdss_sigmas}
\end{figure}

\begin{figure}[p]
	\centering
	\begin{subfigure}{.5\textwidth}
		\centering
		\includegraphics[width=\textwidth]{figures/5_thompson/vstatlas_bl_sigmas}
		\caption{Balanced pool and logistic regression}
		\label{fig:vstatlas_bl_sigmas}
	\end{subfigure}%
	\begin{subfigure}{.5\textwidth}
		\centering
		\includegraphics[width=\linewidth]{figures/5_thompson/vstatlas_br_sigmas}
		\caption{Balanced pool and RBF SVM}
		\label{fig:vstatlas_br_sigmas}
	\end{subfigure}
	\begin{subfigure}{.5\textwidth}
		\centering
		\includegraphics[width=\textwidth]{figures/5_thompson/vstatlas_ul_sigmas}
		\caption{Unbalanced pool and logistic regression}
		\label{fig:vstatlas_ul_sigmas}
	\end{subfigure}%
	\begin{subfigure}{.5\textwidth}
		\centering
		\includegraphics[width=\linewidth]{figures/5_thompson/vstatlas_ur_sigmas}
		\caption{Unbalanced pool and RBF SVM}
		\label{fig:vstatlas_ur_sigmas}
	\end{subfigure}
	\caption[Variance of heuristics (VST ATLAS)]{
		Variance of the expected reward in Thompson sampling with the VST ATLAS dataset.}
	\label{fig:vstatlas_sigmas}
\end{figure}


%%% Local Variables: 
%%% mode: latex
%%% TeX-master: "thesis"
%%% End: 
