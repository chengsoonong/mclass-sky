\documentclass[11pt, oneside]{article}

\usepackage{amsfonts}
\usepackage{amsmath}
\usepackage{amssymb}
\usepackage{geometry}
\usepackage{txfonts}

\geometry{a4paper}

% Define shortcuts for special characters
\newcommand{\bw}{\mathbf{w}}
\newcommand{\bx}{\mathbf{x}}
\newcommand{\by}{\mathbf{y}}
\newcommand{\bbN}{\mathbb{N}}
\newcommand{\bbR}{\mathbb{R}}

\begin{document}
Consider the problem of regression. We are given a function $f : \bbR^d \to \bbR$ for some $d \in \bbN$. We are also given $n$ points $\bx_1, \dots, \bx_n$ for which we know the values. In other words, we know $y_1, \dots, y_n$ where \[
    y_i \coloneqq f(\bx_i)\text{ for }i = 1, \dots, n\text{.}
\]

If we are performing linear regression, then we want to find $\bw$ such that \[
    f(\bx) \approx \bx^T\bw \text{.}
\]

Let $k : \bbR^d \times \bbR^d \to \bbR$ be a kernel. Define the matrix $K$ to be \[
    K \coloneqq \begin{bmatrix}
        K(\bx_1, \bx_1) & \dots & K(\bx_1, \bx_n) \\
        \dots & \dots & \dots \\
        K(\bx_n, \bx_1) & \dots & K(\bx_n, \bx_n)
    \end{bmatrix} \text{.}
\] Define also\[
    \by \coloneqq \begin{bmatrix}
        y_1 \\
        \dots \\
        y_n
    \end{bmatrix} \text.
\]

We can define the error function to be \[
    E(\bw) \coloneqq \frac12 \left(\by - K \bw\right)^T\left(\by - K \bw\right) \text{,}
\] which is the least-squares error. We wish to find $\bw$ that minimises it.

Using differentiation, we can find\[
    \bw = \left(K^T K \right)^{-1} K^T \by \text.
\] For $K$ invertible, this is \[
    \bw = K^{-1}\by\text.
\]

For a given input $\bx$, defining $K_*$ as \[
    K_* = \begin{bmatrix}
        K(\bx, \bx_1) \\
        \dots \\
        K(\bx, \bx_n)
    \end{bmatrix} \text{,}
\]the prediction  is then \[
    y = K_*^T\bw = K_*^T K^{-1}\by \approx f(\bx) \text{.}
\]

This is very similar to Gaussian Process regression, where the prediction is \[
    y = K_*^T\left(K + \sigma^2 I\right)^{-1}\by \text{.}
\]

Kernelised linear regression is then equivalent to Gaussian Process regression when $\sigma = 0$.




\end{document}
