%% Submissions for peer-review must enable line-numbering
%% using the lineno option in the \documentclass command.
%%
%% Preprints and camera-ready submissions do not need
%% line numbers, and should have this option removed.
%%
%% Please note that the line numbering option requires
%% version 1.1 or newer of the wlpeerj.cls file.

\documentclass[fleqn,10pt,lineno]{wlpeerj} % for journal submissions
% \documentclass[fleqn,10pt]{wlpeerj} % for preprint submissions

\usepackage{bm} % bold math symbols

% some convenient symbols
\DeclareMathAlphabet{\mathpzc}{OT1}{pzc}{m}{it}
\DeclareMathOperator{\Beta}{Beta}
\DeclareMathOperator{\Bin}{Bin}
\DeclareMathOperator{\arsinh}{arsinh}
\DeclareMathOperator{\tr}{tr}
\newcommand{\A}{\mathpzc{A}}
\newcommand{\B}{\mathcal{B}}
\newcommand{\X}{\mathcal{X}}
\newcommand{\Y}{\mathcal{Y}}
\newcommand{\Ecal}{\mathcal{E}}
\newcommand{\Normal}{\mathcal{N}}
\newcommand{\Unlabelled}{\mathcal{U}}
\newcommand{\Labelled}{\mathcal{L}}
\newcommand{\R}{\mathcal{R}}
\newcommand*{\argmin}{\operatornamewithlimits{argmin}\limits}
\newcommand*{\argmax}{\operatornamewithlimits{argmax}\limits}


\title{An Empirical Study on Combining Active Learning Suggestions}

\author[1]{Alasdair Tran}
\author[2]{Cheng Soon Ong}
\affil[1]{Data to Decisions CRC, Kent Town, SA 5067, Australia}
\affil[2]{Machine Learning Research Group, Data61, CSIRO, Australia}

\keywords{machine learning, astronomy, active learning, bandit, rank aggregation}

\begin{abstract}
Recent advances in sensors and scientific instruments have led to an increasing use of machine learning techniques for managing the data deluge. Supervised learning has become a widely used paradigm in many big data applications. However,  labeled examples are required during the training phase of supervised machine learning algorithms, and the labeling has become a significant bottleneck. This paper explores the use of machine learning algorithms for identifying informative examples for labeling, the so-called active learning setting. We empirically compare several active learning heuristics on benchmark datasets, and focus on its application to photometric classification of the Sloan Digital Sky Survey. By considering each active learning heuristic as an expert recommendation of which example to label, we propose to combine them using bandit and rank aggregation algorithms. Our results show that combining active learning suggestions improves over each individual heuristic (including passive learning), and provides a promising practical approach.
\end{abstract}

\begin{document}

\flushbottom
\maketitle
\thispagestyle{empty}

\section*{Introduction}

Introduction about active learning.

Definitions and stuff.

\section*{Active Learning Heuristics}
\label{sec:examples}

We shall use four popular heuristics:


Use section and subsection commands to organize your document. \LaTeX{} handles all the formatting and numbering automatically. Use ref and label commands for cross-references.

\subsection*{Figures and Tables}

\begin{figure}[ht]\centering
\includegraphics[width=\linewidth]{figures/vega_filters_and_spectrum}
\caption[Spectrum of the star Vega and the ugriz bandpasses]{The black curve is the                spectrum of Vega, the fifth brightest star in the night sky. The spectrum tells us how                much radiation Vega emits at each wavelength. We also show five transmission functions,                one for each of the five ugriz filters. The transmission function tells us how much light                can get through the filter at each wavelength.}
\label{fig:vega}
\end{figure}

Reference to Figure \ref{fig:vega}.

\begin{table}[h]
	\caption {Summary of active learning heuristics used in our experiments} \label{tab:heuristics}
	\centering
	\begin{tabular}{lll}
		\toprule
		{Name}  & Notation &  Objective  \\
		\midrule
		Entropy & $r_S(\bm{x}; h)$
			& $\argmax_{x \in \Ecal} \left\{-\sum_{y \in \Y} p(y | \bm{x}; h)
            \log \big[ p(y | \bm{x}; h) \big] \right\}$
			\\[2ex]
		Margin & $r_M(\bm{x}; h)$
			& $\argmin_{x \in \Ecal} \left\{ \max_{y \in \Y} p(y | \bm{x}; h) -
            \max_{z \in \Y \setminus \{y\}} p(z | \bm{x}; h)  \right\}$
			\\[2ex]
		QBB Margin & $r_{QM}(\bm{x}; h)$
			& $\argmin_{x \in \Ecal} \left\{ \max_{y \in \Y} p(y | \bm{x}; \B) -
            \max_{z \in \Y \setminus \{y\}} p(z | \bm{x}; \B)  \right\}$
			\\[2ex]
		QBB KL & $r_{QK}(\bm{x}; h)$
			& $\argmax_{x \in \Ecal} \left\{ \dfrac{1}{B}
               \sum_{b=1}^B D_{\mathrm{KL}}(p_b\|p_\B) \right\}$
			\\
		\bottomrule
	\end{tabular}
\end{table}





\begin{table}[ht]
\centering
\begin{tabular}{l|r}
Item & Quantity \\\hline
Widgets & 42 \\
Gadgets & 13
\end{tabular}
\caption{\label{tab:widgets}An example table.}
\end{table}

\subsection*{Citations}

LaTeX formats citations and references automatically using the bibliography records in your .bib file, which you can edit via the project menu. Use the cite command for an inline citation, like \cite{Figueredo:2009dg}, and the citep command for a citation in parentheses \citep{Figueredo:2009dg}.

\subsection*{Mathematics}

\LaTeX{} is great at typesetting mathematics. Let $X_1, X_2, \ldots, X_n$ be a sequence of independent and identically distributed random variables with $\text{E}[X_i] = \mu$ and $\text{Var}[X_i] = \sigma^2 < \infty$, and let
$$S_n = \frac{X_1 + X_2 + \cdots + X_n}{n}
      = \frac{1}{n}\sum_{i}^{n} X_i$$
denote their mean. Then as $n$ approaches infinity, the random variables $\sqrt{n}(S_n - \mu)$ converge in distribution to a normal $\mathcal{N}(0, \sigma^2)$.

\subsection*{Lists}

You can make lists with automatic numbering \dots

\begin{enumerate}[noitemsep]
\item Like this,
\item and like this.
\end{enumerate}
\dots or bullet points \dots
\begin{itemize}[noitemsep]
\item Like this,
\item and like this.
\end{itemize}
\dots or with words and descriptions \dots
\begin{description}
\item[Word] Definition
\item[Concept] Explanation
\item[Idea] Text
\end{description}

We hope you find write\LaTeX\ useful for your PeerJ submission, and please let us know if you have any feedback. Further examples with dummy text are included in the following pages.

\section*{Methods}

\lipsum[4] % Dummy text

\begin{equation}
\cos^3 \theta =\frac{1}{4}\cos\theta+\frac{3}{4}\cos 3\theta
\label{eq:refname2}
\end{equation}

\lipsum[5] % Dummy text

\subsection*{Subsection}

\lipsum[6] % Dummy text

\paragraph{Paragraph} \lipsum[7] % Dummy text
\paragraph{Paragraph} \lipsum[8] % Dummy text

\subsection*{Subsection}

\lipsum[9] % Dummy text


\section*{Results and Discussion}

\lipsum[10] % Dummy text

\subsection*{Subsection}

\lipsum[11] % Dummy text

\subsubsection*{Subsubsection}

\lipsum[12] % Dummy text

\subsubsection*{Subsubsection}

\lipsum[14] % Dummy text

\subsection*{Subsection}

\lipsum[15-20] % Dummy text

\section*{Acknowledgments}

So long and thanks for all the fish.

\bibliography{active}

\end{document}
